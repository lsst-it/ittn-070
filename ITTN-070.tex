\documentclass[PMO,authoryear,lsstdraft,toc]{lsstdoc}


% Package imports go here.

% Local commands go here.

%If you want glossaries
%\input{aglossary.tex}
%\makeglossaries

\title{Rubin Observability Project}

% Optional subtitle
% \setDocSubtitle{A subtitle}

\author{%
Cristian Silva, Joshua Hoblitt
}

\setDocRef{ITTN-070}
\setDocUpstreamLocation{\url{https://github.com/lsst-it/ittn-070}}

\date{\today}

% Optional: name of the document's curator
% \setDocCurator{The Curator of this Document}

\setDocAbstract{%
Scope and requirements of the observability project.
}

% Change history defined here.
% Order: oldest first.
% Fields: VERSION, DATE, DESCRIPTION, OWNER NAME.
% See LPM-51 for version number policy.
\setDocChangeRecord{%
  \addtohist{1}{2023-01-24}{First Version}{Cristian Silva, Joshua Hoblitt}
}


\begin{document}

% Create the title page.
\maketitle
% Frequently for a technote we do not want a title page  uncomment this to remove the title page and changelog.
% use \mkshorttitle to remove the extra pages

% ADD CONTENT HERE
% You can also use the \input command to include several content files.

\section{Introduction}

The Rubin Observatory is a state-of-the-art astronomical facility designed to study the universe through the Legacy Survey of Space and Time (LSST). With its advanced telescopes and cutting-edge technology, the Rubin Observatory is expected to generate an enormous amount of astronomical data that will offer unprecedented insights into the cosmos. However, managing and processing such a massive amount of data presents a significant challenge, and it is essential to ensure the observability of the system to detect, diagnose, and resolve issues that may arise. 

The goal of this document is to set the requirements for an observability project for the Rubin Observatory 

\section{Scope}

The observability project for the Rubin Observatory will cover the following system components:

\begin{itemize}
\item Hardware and software systems supporting the telescope operations.
\item The data processing pipelines used to transform raw telescope data into scientific results
\item The storage and retrieval systems used to store and manage astronomical data
\end{itemize}

\section{Infrastructure}

The following is the summary of the Rubin's Infrastructure

\begin{itemize}
    \item Foreman instances:	4
    \item Puppetserver/puppetdb instances: 4
    \item Hosts known to foreman: 328
    \item Additional servers coming online this year: 100
    \item Network devices (switches, PDUs, etc.): 100
    \item Additional network devices coming this year: 40
    \item Physical locations: 3
    \item Logical sites: 4
    \item K8s clusters: 13
    \item Rook-ceph deployments: 11
    \item Graylog instances: 2
    \item Graylog burst input (per second): 9000
    \item Graylog log volume day (GiB): 205.1
\end{itemize}

Current software status

\begin{itemize}
    \item K8s logs shipped via fluentbit
    \item Host logs via rsyslog
    \item K8s metrics prometheus
    \item Host metrics telegraf
    \item Influx on k8s (per site)
    \item Grafana on k8s (per site)
    \item User authentication on IPA
\end{itemize}  

\section{Requirements}

The following requirements are considered essential:

\begin{itemize}
    \item Provide performance monitoring of hosts and network devices 
    \item Collect log data from hosts and network devices
    \item Collect audit logs from hosts
    \item Trigger notifications based on metrics
    \item Trigger notifications on syslog data
    \item Scrape k8s data. E.g. collect kverno audits
    \item Collect switch counters (snmp, etc)
    \item Collect ceph metrics... e.g. volume data and trigger notifciations
    \item Push notifications via squadcast
    \item Push notifications to a slack channel
    \item Push notifications to a kafka topic
    \item The ability to easily define monitoring checks via config/code in a git repo
    \item Per site observatility/monitoring stack
    \item Dashboards per k8s ns/pod
    \item Dashboards per host and group of hosts
    \item Dashboards for network devices which include per and error counters
    \item Log vew which shows sudo usage. E.g. messages to a slack channel when user sudo's to root
    \item Useful workflows for jr. and helpdesk
    \item Dashboard per group of users on ipa 
    \item Simplified dashboard for non-technical staff
    \item Collect firewall logs
    \item Provide security related alerts. 
\end{itemize} 

The following requirements are considered desirable:

\begin{itemize}
    \item Single dashboard integration of all sites
    \item "Per tentant" k8s workspace metric
    \item "Per tentant" views of hosts logs
    \item "Per tentant" views of pod logs
    \item Basic logs/metrics/monitoring of new hosts to happen automatically (e.g. via foreman or puppetdb)
\end{itemize} 

\appendix
% Include all the relevant bib files.
% https://lsst-texmf.lsst.io/lsstdoc.html#bibliographies
\section{References} \label{sec:bib}
\renewcommand{\refname}{} % Suppress default Bibliography section
\bibliography{local,lsst,lsst-dm,refs_ads,refs,books}

% Make sure lsst-texmf/bin/generateAcronyms.py is in your path
\section{Acronyms} \label{sec:acronyms}
\addtocounter{table}{-1}
\begin{longtable}{p{0.145\textwidth}p{0.8\textwidth}}\hline
\textbf{Acronym} & \textbf{Description}  \\\hline

DM & Data Management \\\hline
\end{longtable}

% If you want glossary uncomment below -- comment out the two lines above
%\printglossaries





\end{document}
